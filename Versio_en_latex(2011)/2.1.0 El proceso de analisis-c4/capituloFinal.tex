
\newpage
\previs[Resumen]

 El an\'alisis de datos de microarrays \'es una disciplina que combina la bioinform\'atica la estad\'istica y la biolog\'ia para esclarecer problemas que aparecen en el estudio de la expresi\'on g\'enica con microarrays, que son herramientas que permiten el estudio de la expresi\'on de manera simult\'anea en todos los genes de un organismo. Con los microarrays se pueden tratar multitud de problemas entre los que podemos destacar la \emph{comparaci\'on de clases}, el \emph{descubrimiento de nuevos grupos} o la construcci\'on de predictores.


%\newpage
%\epileg{Ejercicios de autoevaluaci\'on}

\%begin{preguntes}
%Pregunta 1
%\item Pregunta 1
%\setcounter{numresp}{0}
%\resposta{Respuesta a}
%\resposta{Respuesta b}
%\resposta{Respuesta c}
%\resposta{Respuesta d}

%Pregunta 2
%\item Pregunta 2
%\setcounter{numresp}{0}
%\resposta{Respuesta a}
%\resposta{Respuesta b}
%\resposta{Respuesta c}
%\resposta{Respuesta d}

%Pregunta 3
%\item Pregunta ...
%\setcounter{numresp}{0}
%\resposta{Respuesta a}
%\resposta{Respuesta b}
%\resposta{Respuesta c}
%\resposta{Respuesta d}
%\end{preguntes}

%\newpage
%\epileg{Solucionario}

%\small
%\setlength{\parskip}{10pt}

%\textsbold{1.} d; \textsbold{2.} d; \textsbold{3.} a; ...


%\epileg{Glosario}

%\textbf{Palabra 1} Definici\'on

%\textbf{Palabra 2} Definici\'on

%\textbf{Palabra 3} Definici\'on

%\bibliografia


\bibliographystyle{plain}

\bibliography{MDAreview}



