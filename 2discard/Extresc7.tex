la tabla siguiente que mediante dos ejemplos ilustra las posibles situaciones y forma de resolverlas.


\begin{table}[htbp]
\caption{}
\begin{tabular}{|l|l|l|}
\hline
\multicolumn{1}{|r|}{Tipo de array} & \multicolumn{ 1}{c|}{CDNA  (2 colores)} & \multicolumn{ 1}{c|}{Affy (1 color)} \\ \cline{ 1- 1}
Experimento                       & \multicolumn{ 1}{c|}{} & \multicolumn{ 1}{c|}{} \\ \hline
 &  &  \\
10 individuos & Caso (1, 1):Dise\~no de referencia & Caso (1, 2):Dise\~no comparativo \\
- 5 Diab\'eticos  & 5 arrays Diabet/Referencia & 5 arrays  de diab\'etico \\
- 5 No diab\'eticos   & 5 arrays NoDiabet/Referencia & 5 arrays  de no diab\'etico \\
 (Muestras independientes) &  & \\ \hline
6 individuos & Caso (2, 1) & Caso (2, 2) \\
 & 6 arrays, 1/indiv. & 12 arrays, 2/indiv. \\
- Tejido sano (6)  & Tej. Tumoral/Tej. Sano & 6 Tej. sano, 6 Tej. tumoral \\
- Tejido tumoral  (6) & (Muestras apareadas)  & (Diferencias apareadas ) \\ \hline
\end{tabular}
\label{}
\end{table}

Sin querer profundizar en conceptos te\'oricos, este enfoque contiene posibles puntos problem\'aticos que es
conveniente conocer para evitar cometer errores por mal uso o abuso de los conceptos. Concretamente:
\begin{enumerate}
\item Un p--valor bajo  conlleva a  declarar un gen diferencialmente expresado (la expresi\'on de los genes entre dos muestras difiere de forma significativa) cuando puede ser que no lo est\'e realmente, es decir a pesar de que la probabilidad de que las muestras sean similares sea peque\~na bajo $H_0$, este hecho se podr\'ia haber presentado de forma excepcional. Diremos en este caso que hemos declarado un  \emph{falso positivo}.
\item Los p--valores no son siempre correctos, dado que su validez depende de que se verifiquen ciertas suposiciones sobre los
datos, como la normalidad. Cuando estas suposiciones fallan, los p--valores puedens ser completamente incorrectos.
\end{enumerate}

A continuaci\'on se tratan estos dos aspectos con algo m\'as de detalle:

\subsubsection{El control de las probabilidades de error}

Un test se suele organizar de forma que la probabilidad de falsos positivos est\'e controlada, es decir que sea inferior al nivel
de significaci\'on. Dicho control sin embargo no dice nada de la probabilidad de falsos negativos que puede ser muy alta
sobretodo con peque\~nos tama\~nos muestrales. Es importante no perder de vista la tabla de decisi\'on \ref{tablaDecision} para
recordar que tipos de errores puede conllevar la selecci\'on de genes.

 \begin{center}
%\hspace*{-10em}
\begin{tabular}{l c|c|c|}\label{tablaDecision}
&\multicolumn{3}{c}{\hspace*{3em}DECISI\'ON} \\
&& Aceptar $H_0$ & Rechazar $H_0$ \\ \cline{2-4}
&$H_0$   & Decisi\'on & Error de \\
REALIDAD &cierta  & correcta & TIPO I   \\ \cline{2-4}
&$H_0$   & Error de & Decisi\'on \\
&falsa   & TIPO II  & correcta \\ \cline{2-4}
\end{tabular}
\end{center}
Las probabilidades de cometer un error son
$$
 \underbrace{P(p^+< \alpha |H_0\mbox{ cierta})}_{P(\mbox{Falso positivo (FP))}},\quad
 \underbrace{P(p^*> \alpha|H_0\mbox{ falsa})}_{P(\mbox{Falso negativo (FN))}}.
$$

\subsubsection{Validez de los p--valores y p--valores v\'alidos}
Como se ha dicho, los p--valores dependen de que se verifiquen ciertas suposiciones como la independencia entre observaciones y
normalidad de los datos. En general lo primero suele ser cierto --o asumible-- mientras que lo segundo no tiene porque serlo y
adem\'as es dif\'icil de verificar en cualquier sentido (con 5 o 10 obervaciones, lo que suele ser el caso de muchos estudios de
microarrays no se puede hacer una prueba de normalidad de manera fiable.)

En estos casos se puede proceder de dos formas
\bit
\item Mirar de comprobar gr\'aficamente, de forma directa o indirecta, la hip\'otesis de normalidad.
\item Recurrir a otros tipos de tests, como tests no param\'etricos o tests de permutaciones que no precisan de la suposici\'on
de normalidad.
\eit
En general, dado que tanto los tests de permutaciones como los no param\'etricos requieren de taman\~os muestrales considerables, se suele utilizar distintas variantes del test--t, como las que se han discutido en la secci\'on anterior.



setwd(workingDir)
> stopifnot(require(Biobase)) #library(Biobase)
> load (file=file.path(dataDir, "celltypes-normalized.rma.Rda"))
> my.eset <- eset_rma_filtered
> grupo_1 <- as.factor(pData(my.eset)$treat)
sapply(teststat, function(x){print(x[1:5])})

(https://
caarraydb.nci.nih.gov/caarray/performExperimentSearchAction.do
